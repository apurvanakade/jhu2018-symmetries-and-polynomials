%!TEX root = index.tex

\title{Symmetries \& Polynomials}
\author{\small{Apurva Nakade}}
\date{}
\maketitle
\section{Introduction}

A very natural question that one comes across very frequently in mathematics is: 
  \begin{quote}
    \textbf{Q.} Given a polynomial $ p(x)$ find (all of) it's roots.
  \end{quote}
Mathematicians are especially interested in finding solutions which can be described using only algebraic operations like arithmetic operations and radicals $ \sqrt[n]{-}$. \\

When $ p(x)$ has degree 2 this is elementary. When $ p(x)$ has degree 3 or 4 the algebra is more complicated but still quite elementary. In practice it might not be possible to do this by hand but you could, for example, write a program to do it for you.\\

It then came as a big surprise then when Abel-Ruffini showed that there exists polynomials of degree 5 (quintic) which cannot be solved using radicals! Galois extended the theory further and explained the underlying reason using symmetry groups. (The wikipedia article on Galois theory is worth reading and it describes the interesting historical developments quite well.)\\

In this course we'll look at the solutions of a cubic (degree 3) and a quartic (degree 4) using methods inspired from Galois theory and then provide an introduction to the underlying theory. Galois' work as we understand it today established a connection, called \emph{Galois correspondence}, between two seemingly unrelated mathematical structures called \emph{fields} and \emph{groups}. The ``easier'' of these two is the theory of \emph{symmetry groups} which we'll explore as much or as little as time permits. There are no well-defined goals in this course.\\\\

\noindent \textbf{How to use these notes:}
These notes are meant for students who do not have any knowledge of group theory. Not everything in these notes is mathematically correct but the errors are there to minimize the length and keep the notes elementary.

Each page contains problems that you should solve before moving on to the next as the following pages sometimes contain answers to the questions on the previous pages. 